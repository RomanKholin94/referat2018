\documentclass[a4paper,12pt]{article}
\usepackage[utf8x]{inputenc}
\usepackage[russian]{babel}
\usepackage[pdf]{graphviz}
\usepackage{float}
\usepackage{amsmath}

\begin{document}
Пусть даны два графа $a, b$; $deg(a), deg(b)\le 2$. Каждому ребру присвоенно имя. Даны следующие операции:\newline
a)Двойная переклейка\newline
\begin{figure}[H]
\begin{minipage}[h]{0.49\linewidth}
\digraph[scale=0.7]{g1}{
    rankdir=LR;
    node [shape="point"];
    d->e [dir=none, color=red, label="i'"];
    e->f [dir=none, color=red, label="j'"];
    a->b [dir=none, color=blue, label="i"];
    b->c [dir=none, color=blue, label="j"];
}
\caption{До cклейки}
\end{minipage}
\begin{minipage}[h]{0.49\linewidth}
\digraph[scale=0.7]{g2}{
    rankdir=LR;
    node [shape="point"];
    d->e [dir=none, color=red, label="i'"];
    e->f [dir=none, color=blue, label="j"];
    a->b [dir=none, color=blue, label="i"];
    b->c [dir=none, color=red, label="j'"];
}
\caption{После cклейки}
\end{minipage}
\end{figure}
b)Полуторная переклейка\newline
\begin{figure}[H]
\begin{minipage}[h]{0.49\linewidth}
\digraph[scale=0.7]{g3}{
    rankdir=LR;
    node [shape="point"];
    d->e [dir=none, color=red, label="i'"];
    a->b [dir=none, color=blue, label="i"];
    b->c [dir=none, color=blue, label="j"];
}
\caption{До cклейки}
\end{minipage}
\begin{minipage}[h]{0.49\linewidth}
\digraph[scale=0.7]{g4}{
    rankdir=LR;
    node [shape="point"];
    d->e [dir=none, color=red, label="i'"];
    e->f [dir=none, color=blue, label="j"];
    a->b [dir=none, color=blue, label="i"];
}
\caption{После cклейки}
\end{minipage}
\end{figure}
c)Разрез\newline
\begin{figure}[H]
\begin{minipage}[h]{0.49\linewidth}
\digraph[scale=0.7]{g5}{
    rankdir=LR;
    node [shape="point"];
    b->c [dir=none, label="i'"];
    a->b [dir=none, label="i"];
}
\caption{До разреза}
\end{minipage}
\begin{minipage}[h]{0.49\linewidth}
\digraph[scale=0.7]{g6}{
    rankdir=LR;
    node [shape="point"];
    c->d [dir=none, label="i'"];
    a->b [dir=none, label="i"];
}
\caption{После разреза}
\end{minipage}
\end{figure}
d)Склейка\newline
\begin{figure}[H]
\begin{minipage}[h]{0.49\linewidth}
\digraph[scale=0.7]{g7}{
    rankdir=LR;
    node [shape="point"];
    c->d [dir=none, label="i'"];
    a->b [dir=none, label="i"];
}
\caption{До cклейки}
\end{minipage}
\begin{minipage}[h]{0.49\linewidth}
\digraph[scale=0.7]{g8}{
    rankdir=LR;
    node [shape="point"];
    b->c [dir=none, label="i'"];
    a->b [dir=none, label="i"];
}
\caption{После cклейки}
\end{minipage}
\end{figure}
e)Удаление\newline
\begin{figure}[H]
\begin{minipage}[h]{0.49\linewidth}
\digraph[scale=0.7]{g9}{
    rankdir=LR;
    node [shape="point"];
    e->f [dir=none, label="i'"];
    d->e [dir=none, color=red, label="q"];
    c->d [dir=none, color=red, label="..."];
    b->c [dir=none, color=red, label="p"];
    a->b [dir=none, label="i"];
}
\caption{До удаления}
\end{minipage}
\begin{minipage}[h]{0.49\linewidth}
\digraph[scale=0.7]{g10}{
    rankdir=LR;
    node [shape="point"];
    b->c [dir=none, label="i'"];
    a->b [dir=none, label="i"];
}
\caption{После удаления}
\end{minipage}
\end{figure}
f)Вставка\newline
\begin{figure}[H]
\begin{minipage}[h]{0.49\linewidth}
\digraph[scale=0.7]{g11}{
    rankdir=LR;
    node [shape="point"];
    b->c [dir=none, label="i'"];
    a->b [dir=none, label="i"];
}
\caption{До вставки}
\end{minipage}
\begin{minipage}[h]{0.49\linewidth}
\digraph[scale=0.7]{g12}{
    rankdir=LR;
    node [shape="point"];
    e->f [dir=none, label="i'"];
    d->e [dir=none, color=blue, label="q"];
    c->d [dir=none, color=blue, label="..."];
    b->c [dir=none, color=blue, label="p"];
    a->b [dir=none, label="i"];
}
\caption{После вставки}
\end{minipage}
\end{figure}
У каждой операции есть своя цена. Требуется найти такую последовательность операций, преобразующую структуру $a$ в структуру $b$,
такую, что сумма операций минимальна.\newline
Общм графом назовём...\newline

\end{document}
