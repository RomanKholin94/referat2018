\documentclass[a4paper,12pt]{article}
\usepackage[utf8x]{inputenc}
\usepackage[russian]{babel}
\usepackage[pdf]{graphviz}
\usepackage{float}
\usepackage{amsmath}

\begin{document}
-Назовём ориентированный граф степенями вершины меньше $2$ структрой. Все реюра имеют имена. Ребра графа будем
называть дугами, края дуг также имеют метку: начало дуги будем помечать тем же именем, что и дугу, но с индексом
$1$, конец дуги - с индексом $2$.\newline
-Над структурой можно выпонять следующие оперции:\newline
-Каждая оперция имеет цену.\newline
-Задача состоит в следующем: требуеться найти кратчайшую последовательность операций (т.е. имеющую минимальную
сумарную цену), преобразующую структуру $a$ в структуру $b$.\newline
-Пусть даны две структуры $a$ и $b$. Для них существует общим граф($a+b$), который строиться взаимооднозначно со
структурами:\newline
-Обычная вершина - вершина, соответствующая краю дуги, представленному и в $a$ и в $b$.
-Обычное ребро соединяет две обычные вершины, если они соеденены или в $a$ (тогда это ребро помечаеться пометкой
$a$) или в $b$ (если два края дук инцинденты в обоих графах, то врешины соеденины двумя ребрами, с пометками
$a$ и $b$ соответственно).\newline
-Особая вершина - вершина особого графа, соответствующая максимальному по включению связанному подграфу одной
из структур.\newline
-Особое ребро - ребро, связывающая особую и обычную вершины.\newline
-Операции над общим графом:\newline
 -Удаление особой вершины - удалить особую вершину и объеденить два инцентных ей особых ребра одному обычному.
Удаление особой вершины с пометкой $a$ соответствует удалению участка графа из структуры $a$, а с пометкой $b$ -
вствку участка графа в стркуктуру $b$.\newline
 -Двойная переклейка - удалить два одинаково помеченных ребра общего графа и соединить четыре образовавшихся
конца двумя новыми неинцидентными ребрами с той же пометкой.\newline
 -Вырезание обычного ребра(например, с пометкой $a$) - из цикла строго больше двух заменить два соседних
$b-$ребра на два других $b-$ребра - одно из них образует финальный $2-$цикл, второе соединяет два дальних
конца.\newline
 -Инверсия - двойная переклейка с изменением порядка рёбер.\newline
-Теорема. В циклическом случае нижеуказанный алгоритм строит последовательность операций с минимальной суммарной
ценой. Его время и используемая им память линейно зависят от суммарного размера двух исходных структур.\newline
-Докозательство. Рассмотрим три случая:\newline
1) $0<w\le1$\newline
Пусть $B$ - количество особых вершин в нём графе.\newline
Отрезок - максимальный по свзанности подграф, состоящий из обычных ребер.\newline
$X$ - множество нециклических отрезков.\newline
$S1$ = $\sum_{x\in X}{[\frac{|x|}{2}]}$\newline
$S2$ - число циклических отрезков.\newline
$S = S1 - S2$\newline
$C$ - качество графа(количество циклов без петель).\newline
$U1$ - особые петли.\newline
$U2$ - особые вершины(не петли)\newline
$U = U1 + U2$\newline
$Q$ - количесвто двойных переклеек, выпоненых в следующем алгоритме.\newline
Опишем алгоритм, приводящий граф к финальному виду:\newline
1)Удаляем все петли (всего $U1$ раз).\newline
2)Вырезаем все обычные ребра (всего $S$ раз)\newline
3)К этому моменту остаються только циклы с особыми ребрами. Разбиваем такие циклы на $2-$циклы, а именно: пусть
речь идет об участке $abba$. Двойной переклейкой удалються $a-$ соседи особой вершины и заменяються на два
ребра так, что дающих два других цикла, один из кторых - $abb$. В этом цикле
удаляеться особая вершина и получаеться $2-$цикл финального вида. Большой цикл разбиваеться аналогичным
образом. Итого, выполняется $Q$ раз производится двойная переклейка и $U2$ раз - удаление особой
вершины.\newline
Пусть $T(G)$ - сумарная цена операций.\newline
$B + S = U + Q$\newline
$U = B + S - Q$\newline
Суммарная стоимость получается $Q + wU = (1 - w)Q + w(B + S)$\newline
Выразим $Q$. Пусть $D$ - сумарный размер компонент графа. Его финальный вид состоит из $\frac{D}{2}$ финальных
циклов. Вычитая из качества конечного графа качество начального, получаем $Q = \frac{D}{2} - C$.\newline
Итого, $T(G) = (1 - w)(\frac{D}{2} - C) + w(B + S)$\newline
 -1<w<=2\newline
 -2<w\newline
\end{document}
