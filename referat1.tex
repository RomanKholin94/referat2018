\documentclass[a4paper,12pt]{article}
\usepackage[utf8x]{inputenc}
\usepackage[russian]{babel}
\usepackage[pdf]{graphviz}
\usepackage{float}
\usepackage{amsmath}
\usepackage[usenames]{color}

\begin{document}
\begin{figure}[H]
\begin{minipage}[h]{0.49\linewidth}
\digraph[scale=0.7]{g13}{
    rankdir=LR;
    node [shape="circle"];
    a [label=<1<SUB>1</SUB>3<SUB>2</SUB>>];
    b [label=<1<SUB>2</SUB>2<SUB>1</SUB>>];
    c [label=<2<SUB>2</SUB>3<SUB>1</SUB>>];
    d [label=<6<SUB>1</SUB>>];
    e [label=<6<SUB>2</SUB>>];
    f [label=<7<SUB>1</SUB>>];
    g [label=<7<SUB>2</SUB>8<SUB>1</SUB>>];
    h [label=<8<SUB>2</SUB>>];
    i [label=<9<SUB>1</SUB>>];
    j [label=<9<SUB>2</SUB>10<SUB>1</SUB>>];
    k [label=<10<SUB>2</SUB>>];

    c->a [dir=forward, label="3"];
    b->c [dir=forward, label="2"];
    a->b [dir=forward, label="1"];
    d->e [dir=forward, label="6"];
    f->g [dir=forward, label="7"];
    g->h [dir=forward, label="8"];
    i->j [dir=forward, label="9"];
    j->k [dir=forward, label="10"];
}
\caption{Граф a}
\end{minipage}
\begin{minipage}[h]{0.49\linewidth}
\digraph[scale=0.7]{g14}{
    rankdir=LR;
    node [shape="circle"];
    a [label=<1<SUB>1</SUB>4<SUB>2</SUB>>];
    b [label=<1<SUB>2</SUB>2<SUB>1</SUB>>];
    c [label=<2<SUB>2</SUB>4<SUB>1</SUB>>];
    d [label=<5<SUB>1</SUB>5<SUB>2</SUB>>];
    e [label=<7<SUB>1</SUB>>];
    f [label=<7<SUB>2</SUB>>];
    g [label=<9<SUB>1</SUB>>];
    h [label=<9<SUB>2</SUB>>];
    i [label=<10<SUB>1</SUB>>];
    j [label=<10<SUB>2</SUB>>];

    c->a [dir=forward, label="4"];
    b->c [dir=forward, label="2"];
    a->b [dir=forward, label="1"];
    d->d [dir=forward, label="5"];
    e->f [dir=forward, label="7"];
    g->h [dir=forward, label="9"];
    i->j [dir=forward, label="10"];
}
\caption{Граф a}
\end{minipage}
\end{figure}
\begin{figure}[H]
\begin{minipage}[h]{0.49\linewidth}
\digraph[scale=0.7]{g15}{
    rankdir=LR;
    node [shape="circle"];
    a1 [label=<1<SUB>1</SUB>>, color="green"];
    a2 [label=<1<SUB>2</SUB>>, color="green"];
    b1 [label=<2<SUB>2</SUB>>, color="green"];
    b2 [label=<2<SUB>1</SUB>>, color="green"];
    a3 [label=<3a>, color="red"];
    b4 [label=<4b>, color="red"];
    a6 [label=<6a>, color="red"];
    b5 [label=<5b>, color="red"];
    a7 [label=<7<SUB>1</SUB>>, color="green"];
    b7 [label=<7<SUB>2</SUB>>, color="green"];
    a8 [label=<8a>, color="red"];
    a9 [label=<9<SUB>1</SUB>>, color="green"];
    b9 [label=<9<SUB>2</SUB>>, color="green"];
    a10 [label=<10<SUB>1</SUB>>, color="green"];
    b10 [label=<10<SUB>2</SUB>>, color="green"];
    
    b1->a2 [dir=none, color=green, label="a"];
    b1->a2 [dir=none, color=green, label="b"];
    a1->a3 [dir=none, color=red, label="a"];
    a3->b2 [dir=none, color=red, label="a"];
    b2->b4 [dir=none, color=red, label="b"];
    b4->a1 [dir=none, color=red, label="b"];
    b5->b5 [dir=none, color=red, label="b"];
    b7->a8 [dir=none, color=red, label="a"];
    b9->a10 [dir=none, color=green, label="a"];
}
\caption{Общий граф a+b}
\end{minipage}
\end{figure}
\textcolor{green}{Обычная вершина}\newline
\textcolor{red}{Особая вершина}\newline
\textcolor{green}{Обычное ребро}\newline
\textcolor{red}{Особое ребро}\newline
Блок - максимальные по включению связанные участки вершин, принадлежащие одной структуре.\newline
Петля - 5b (т.е. цикл, который является блоком)\newline
Висячее ребро - ребро, инцидентное особой вершине степени 1\newline
Невисячие особые ребра присутствуют в нем парами — ребрами, инцидентными одной особой вершине; такую пару удобно считать за одно двойное 
ребро.\newline

\end{document}
